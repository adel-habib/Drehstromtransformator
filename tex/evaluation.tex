
\subsection{Experimental setup}



\lipsum
\subsection{Results}
The results of your experiments. Compare different variants of your design (e.g.~with and without optimizations) or compare performance to other designs or systems. Plots should show the average over multiple runs (at least 10 as a rule of thumb), including error bars, percentiles or min/max values.

Discuss the plot and extract the overall performance. Do not repeat all numbers in the text, but mention relevant differences in numbers, e.g.~our optimization improves throughput by 26\%.
Discuss how the results validate or contradict your assumptions.

Perform experiments to evaluate your system under operating normal conditions, when experiencing failures or attacks, or with different workloads.

%TODO LEERLAUF VEERSUCH 

% LEERLAUF U_1L vs U_2L
\begin{figure}
	\begin{tikzpicture}
		\begin{axis}[
				/pgf/number format/.cd,
				use comma,
				1000 sep={},
				xlabel=$U_{1L}$,
				ylabel=$U_{2L}$,
				%legend entries={baseline, optimized},
			]

			\addplot table [x=U_1L, y=U_2L, col sep=comma] {data/leerlauf.csv};
			%\addplot table [x=U_1L, y=ue, col sep=comma] {data/leerlauf.csv};
		\end{axis}
	\end{tikzpicture}
	\caption{A graph showing latency and throughput of a baseline and optimized implementation. The axes show latency in milliseconds, and throughput in thousand operations per second. Data is made up.}
\end{figure}

% LEERLAUF U_1L vs Ü
\begin{figure}
	\begin{tikzpicture}
		\begin{axis}[
				/pgf/number format/.cd,
				use comma,
				1000 sep={},
				xlabel=$U_{1L}$,
				ylabel=Ü,
				%legend entries={baseline, optimized},
			]
			%The mockup experiment data is stored in a csv file, and imported here.
			%\addplot table [x=U_1L, y=U_2L, col sep=comma] {data/leerlauf.csv};
			\addplot table [x=U_1L, y=ue, col sep=comma] {data/leerlauf.csv};
		\end{axis}
	\end{tikzpicture}
	\caption{A graph showing latency and throughput of a baseline and optimized implementation. The axes show latency in milliseconds, and throughput in thousand operations per second. Data is made up.}
\end{figure}


%TODO KS VEERSUCH 

% KS U_1kl und I_2kl vs U_2kl
\begin{figure}
	\begin{tikzpicture}
		\begin{axis}[
				/pgf/number format/.cd,
				use comma,
				1000 sep={},
				xlabel=$I_{2kL}$ (A),
				ylabel={$ U_{1kL} (\mathrm{V}), \quad I_{2kL}$ (A)},
				legend entries={$U_{1kL}$, $I_{2kL}$},
				legend pos=north west,
			]
			%The mockup experiment data is stored in a csv file, and imported here.
			\addplot table [x=I_2kl, y=U_1kl, col sep=comma] {data/kurzschlussversuch.csv};
			\addplot table [x=I_2kl, y=I_2kl, col sep=comma] {data/kurzschlussversuch.csv};
		\end{axis}
	\end{tikzpicture}
	\caption{A graph showing latency and throughput of a baseline and optimized implementation. The axes show latency in milliseconds, and throughput in thousand operations per second. Data is made up.}
\end{figure}

%TODO R VEERSUCH 
%%% 

% U_2L vs I_2L 
\begin{figure}
	\begin{tikzpicture}
		\begin{axis}[
				/pgf/number format/.cd,
				use comma,
				1000 sep={},
				xlabel=$I_{2L}$ (A),
				ylabel={$ U_{2L} (\mathrm{V})$},
			]
			\addplot table [x=I_2L (A), y=U_2L, col sep=comma] {data/ohmsche_belastung.csv};
		\end{axis}
	\end{tikzpicture}
	\caption{A graph showing $U_2L$ vs $I_2L$.}
\end{figure}

% Eta vs I_2l 
\begin{figure}
	\begin{tikzpicture}
		\begin{axis}[
				/pgf/number format/.cd,
				use comma,
				1000 sep={},
				xlabel=$I_{2kL}$ (A),
				ylabel={$ U_{1kL} (\mathrm{V}), \quad I_{2kL}$ (A)},
			]
			%The mockup experiment data is stored in a csv file, and imported here.
			\addplot table [x=I_2L (A), y=eta, col sep=comma] {data/ohmsche_belastung.csv};
			%\addplot table [x=I_2kl, y=I_2kl, col sep=comma] {data/ohmsche_belastung.csv};
		\end{axis}
	\end{tikzpicture}
	\caption{A graph showing $ \eta $ vs $I_2L$.}
\end{figure}

% POWER FACTOR VS I_2L
\begin{figure}
	\begin{tikzpicture}
		\begin{axis}[
				/pgf/number format/.cd,
				use comma,
				1000 sep={},
				xlabel=$I_{2kL}$ (A),
				ylabel={$ U_{1kL} (\mathrm{V}), \quad I_{2kL}$ (A)},
			]
			%The mockup experiment data is stored in a csv file, and imported here.
			\addplot table [x=I_2L (A), y=PF, col sep=comma] {data/ohmsche_belastung.csv};
			%\addplot table [x=I_2kl, y=I_2kl, col sep=comma] {data/ohmsche_belastung.csv};
		\end{axis}
	\end{tikzpicture}
	\caption{A graph showing power factor vs, $I_{2L}$.}
\end{figure}

\begin{figure}
	\begin{tikzpicture}
		\begin{axis}[
				/pgf/number format/.cd,
				use comma,
				1000 sep={},
				xlabel=$I_{2kL}$ (A),
				ylabel={$ U_{1kL} (\mathrm{V}), \quad I_{2kL}$ (A)},
			]
			%The mockup experiment data is stored in a csv file, and imported here.
			\addplot table [x=I_2L (A), y=I_1L, col sep=comma] {data/ohmsche_belastung.csv};
			%\addplot table [x=I_2kl, y=I_2kl, col sep=comma] {data/ohmsche_belastung.csv};
		\end{axis}
	\end{tikzpicture}
	\caption{A graph showing $I_{1L}$ vs, $I_{2L}$.}
\end{figure}


%TODO C VEERSUCH 
%%% 


%TODO L VEERSUCH 
%%% 
\lipsum

